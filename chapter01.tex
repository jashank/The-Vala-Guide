\chapter{Introducing Vala}

Welcome to \emph{The Vala Guide}!  This is intended as an introduction to the
Vala programming language, the ``next-generation'' language of the GNU/Linux
desktop.  It's object-oriented and powerful, but runs with native speed, and
is cross-platform.

\section{Another Programming Language?}

A lot of the application software that runs on *nix desktops is written in C,
leveraging the \emph{GLib} library, a bundle of useful utilities that
implement a range of powerful programming primitives (akin to those provided
by APR or Boost).  Naturally, it has bindings into a range of other
programming languages, but one of the most powerful features is called
GObject: an object-oriented layer using pure C, allowing the manipulation of
complex, abstract data structures.

C, however, is hard.  Having to manipulate objects in a language that isn't
really designed for it is not a pleasant experience, and most ports of
GLib-based libraries (including Gtk+, a popular widget and toolkit library) to
true object-oriented languages incur runtime overheads: most languages
(including C++ and Objective-C) require some form of `name mangling' which
must be handled by a language runtime with a performance penalty.  And the
Java approach---completely decoupling the language from platform-dependent
binaries, instead targeting a MIPS-like architecture which is interpreted by
the Java virtual machine, a simulation of that architecture that translates
into native machine code at runtime---incurs even greater performance
penalties\footnote{Although, in many cases, Java applications will often run
  nearly as fast as their native counterparts because of the huge optimisation
  efforts that have gone into the development of the reference Java virtual
  machine.}.

Yet, all the groundwork existed for a language to be implemented atop the GLib
and GObject framework, and the potential for such a language, featuring a
truly object-oriented syntax, was great.  And thus, Vala, created by J\"u{}rg
Billeter and Raffaele Sandrini in mid-2006, was born.  A programming language
more familiar to developers with a grounding in C\# or Java, Vala provides a
lot of features found in other high level programming languages in a manner
that is familiar, concise and easy to use, without introducing a language
runtime or virtual machine to slap on a performance penalty.  To achieve this,
the Vala compiler does a surprising thing: it generates (surprisingly legible)
C, which can be fed into any C compiler and linked against the GLib libraries
(and any additional libraries that are required) to produce a native
executable.

\section{What Can Vala Do?}

Vala gives you, the developer, a familiar, object-oriented syntax connected to
a wide range of libraries out of the box.  It features mixins/interfaces,
properties, signals, lambdas, pleasant \code{foreach} loops, type inference,
generics, nullables, reference-counting memory management and exception
handling.

Vala does not introduce any feature that cannot be accomplished using C with
GLib, but it makes using these features much simpler.  One of the most common
criticisms of GLib is that it is exceptionally verbose, and requires lots of
boilerplate code, which, since you're using Vala, you don't have to worry
about: the Vala compiler writes it for you.

And it's not a pain to use other libraries, either: any library that your C
compiler can link against, no matter what language it's written in, Vala can
work with, particularly if it uses GObject.  All that's required is a
\code{vapi} file, which serves a similar purpose to a conventional C,
Objective-C or C++ header, unlike other languages, where specially-created
bindings and foreign-function interfaces are required.

Programs written in C can also make use of libraries written in Vala without
any additional effort whatsoever. The vala compiler will automatically
generate the header files required, which C programmes need to use the
library. It is also possible to write bindings for Vala from other languages
like python and Mono.

\section{Vala and Genie}

If you don't like Vala's C\#-like syntax, the Vala compiler infrastructure
also supports another language, called Genie, which serves a very similar
purpose.  It has most, if not all, of the same features that Vala has, and
uses the same mechanisms to achieve it's tasks, with one major difference: it
looks a lot more like Python.

Genie is (currently) beyond the scope of this guide, but the concepts in Vala
are sufficiently similar that, if you're learning Genie, you will find this
guide useful.

More information about Vala is available at

\hspace{0.25in}\url{http://live.gnome.org/Vala/}

\section{In This Guide}

XXX to be written

\section{Conventions}

XXX to be determined (o\_O)

\section{Getting Vala and Friends}

If you're on any modern *nix system, you're in luck.  All major GNU/Linuxen
(including Ubuntu, Fedora, Debian, Gentoo and Beyond Linux From Scratch)
support Vala development, as do FreeBSD, Darwin, Mac OS X and all OSes
supported by \emph{pkgsrc}, amongst others.  Your preferred package manager
should give you the Vala toolchain (and the necessary tools, including a C
compiler, and the GLib libraries, if you don't already have them installed) if
you install \code{vala}, which may be categorised into `languages' or similar.

The GNOME website has full instructions for installing Vala if your
distribution is, for some weird reason, not yet on the Vala bandwagon.  Let
your distribution know that they're missing, out, too!
