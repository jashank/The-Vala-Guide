\chapter{Vala Basics}

This chapter briefly introduces some of the basic features of Vala.  Ready?

\section{Hello, Vala!}

Let's begin at the place where every other language reference begins: with the
classic ``hello world'' example.

In your favourite text editor\footnote{It's probably a worthwhile idea to
  investigate one of the Vala IDEs; Geany (not to be confused with Genie) and
  MonoDevelop both have excellent Vala facilities.  However, I (Jashank) use
  GNU Emacs for all my code.}, create a file called \code{helloworld.vala},
containing the following:

\inputvalacodefile{src/chapter02/helloworld.vala}

Vala is case-sensitive.  And seasoned Java programmers should resist the urge
to name the file \code{HelloWorld.vala}; there's no file naming scheme
necessary in Vala.

\subsection{Compiling `helloworld'}

As has previously been mentioned, Vala is compiled to C, and this is done by a
tool called \func{valac}.  In a terminal, change to the directory you saved
\code{helloworld.vala} to, and run

\begin{bashcommands}
> valac helloworld.vala
\end{bashcommands}

All proceeding normally, you should see no errors---you may see warnings from
the GLib type system, GType, but these are (currently) harmless.  However, a
new executable, \func{helloworld}, will now be present in that directory, and
you can run it from the terminal.

\begin{bashcommands}
> ./helloworld
Hello, world!
\end{bashcommands}

If you're feeling morbidly curious, you can ask Vala for the C source that
it's generated by running \func{valac -C}.

\subsection{Stepping through `helloworld'}

Small though this example is, it's complete enough to show us a number of
features of Vala.

For starters, Vala features both C-style multiline comments, delimited by
\inlinecode{/* ... */}, and the C++-style single-line comments, denoted by
\inlinecode{//} and running to the end of the line.

Vala also follows the convention of requiring a function called \code{main},
which is the execution entry point, as in most C-like languages.  Any
executable has the choice of returning an integer exit status to the operating
system; in this case, we choose not to, and thus our \code{main()} function
returns \code{void}.

\inputvalafilesection{src/chapter02/helloworld.vala}{7}{7}{0}

Like many C-like languages, all functions have a ``prototype'', which tells
the compiler how many arguments of what type are expected, and what the
function intends to return.  The first part shows the return type (here,
\code{void}) and can optionally show a selection of modifiers, which we'll
come to shortly.  Next, we have the name of the function, followed by a
parenthesised list of type and name pairs, signalling the arguments that the
function expects, and finally, we open the block containing the function's
code.

The last character on the line is the \inlinecode{\{}. This signals the start
of \inlinecode{main( )}’s body. All of the code that comprises a method will
occur between the method’s opening curly brace and its closing curly brace.

One other point: \inlinecode{main( )} is simply a starting place for your
programme. A complex programme will have dozens of classes, only one of which
will need to have a \inlinecode{main( )} method to get things started.

\inputvalafilesection{src/chapter02/helloworld.vala}{8}{8}{4}

This line is a \emph{single-line comment}, the other type of comment supported by Vala. A single-line comment commences with the character sequence \inlinecode{//} and terminates at the end of the line. They are generally used to add short explanations to particular lines of code, while multi-line  comments are used to add longer explanations.

\inputvalafilesection{src/chapter02/helloworld.vala}{9}{9}{4}

This line outputs the string of characters “Hello, World” to the screen, followed by a new-line character, that moves subsequent output to the next line. The \inlinecode{printf( )} method shown here is part of GLib a library that Vala relies upon extensively. The \inlinecode{printf} method will print the text inside the parenthesis to the screen. The \inlinecode{\textbackslash n} is a special character sequence that represents a newline.

Notice that the \inlinecode{printf( )} statement ends with a semicolon. All statements in vala end with a semicolon. The only statement in this programme is the \inlinecode{printf( )} statement. All the other lines are not statements - they are either comments, method declarations, or in the case of lines with only the \inlinecode{\{} or \inlinecode{\}} characters, scope descriptors.

\section{Programming with Variables}

Although the programme written above instructs the computer, it is not very useful as everything in the programme was static. The programme would produce the same output on every run, and was not doing any calculation. Fundamental to making programmes more dynamic is the concept of a variable. A variable is just a name for a certain memory location. It can be thought of as a box where you can store data. The the data in a variable can be modified as a programme is running, thus allowing for a programme to become dynamic.

This next programme introduces a few basic types of variables that are available in Vala.

\inputvalacodefile{src/chapter02/variable_example.vala}

You can compile and run this programme by typing

\begin{bashcommands}
$ valac variable_example.vala
$ ./variable_example
\end{bashcommands}

The output will look like this.

\begin{bashcommands}
false
v
10
15.33
Varun Madiath
20
\end{bashcommands}

Let's now dissect this code to understand how this output was generated.

\inputvalafilesection{src/chapter02/variable_example.vala}{9}{13}{4}

These lines create variables, and assign them a value. There are different types of variables being created here. The \inlinecode{bool} variable represents a boolean, or a variable which has only two possible values, \inlinecode{true} and \inlinecode{false}. The unichar variable represents a character of text, so almost any character can be stored in this variable (yes that includes characters from Japanese and Indic scripts). The character is enclosed in single-quation marks to indicate that it is a character, and not an command for the computer. The \inlinecode{int} variable represents an integer (a number which doesn't have a decimal point in it). The double variable represents a floating point number (a variable which may or may not have a decimal point in it) and finally a \inlinecode{string} variable represents a series (or string) of characters. A string is enlosed in double quotation marks to demarcate the text as a string.

Variables are assigned values with the \inlinecode{=} operator, with the name and type of the operator on the left side of the operator, and the value on the right side. The general form of a variable declaration can thus be written as (text inside square brackets is optional).

%TODO change this to a vala insert
\begin{bashcommands}
type name [= value]
\end{bashcommands}

\inputvalafilesection{src/chapter02/variable_example.vala}{16}{20}{4}

These lines print the values of the variables to the screen. Printing to the screen is done using the same \inlinecode{printf()} method that was used in the first example. You will notice that inside the \inlinecode{printf()} method most variable names are appended with \inlinecode{.to\_string()}. This is because the \inlinecode{printf()} method must be given a string to print to the screen. This is also the reason why the string variable is not appended with \inlinecode{.to\_string()}.

\inputvalafilesection{src/chapter02/variable_example.vala}{23}{23}{4}

This line modifies the value of \inlinecode{num} by assigning it a new value. The new value is given by the expression \inlinecode{num * 2}. It is important to not look at this as an algebraic expression, as that would not make sense, but rather as an assignment. There is another operator for equality, that we will discuss later. Here the value of \inlinecode{num} is multiplied by two, and the result is stored in \inlinecode{num}.

\inputvalafilesection{src/chapter02/variable_example.vala}{26}{26}{4}

This line prints the value of \inlinecode{num}. As you can see in the output, the value has changed after the assignment above.

\section{A runthrough of some basic Vala Grammar}
We have now written two small programmes in Vala, but before we proceed any further it would be wise to have a look at the various elements that make a programme. The different elements that compose a Vala programme are:

\begin{itemize}
\item Whitespace
\item Comments
\item Identifiers
\item Keywords
\item Literals
\item Operators
\item Separators
\end{itemize}

\section{Whitespace}
Vala is a free-form language. A free-form language is one where code is not required to be formatted in any particular fashion. The programmes that we have written till now could be condensed into a single line (with the exception of the single line comments). The only requirement is that every token is separated by at least one character of whitespace, unless it is already separated by a separator or operator. Separators and operators will be discussed in just a moment.

\section{Comments}
We have already introduced the two types of comments that Vala supports. The single-line comment and the multi-line comment.

It is important to note that the sequence of characters */ cannot appear anywhere in your comment, as the compiler will understand that character sequence to be the end of the comment. This implies that one inline comment may not be placed inside another. As you begin to write longer, more complicated programmes, you might wish to comment out a section of your code in order to try tracking the source of an error. In such cases uses a multi-line comment, is inadvisable unless you specifically know that there is no multi-line comment in that block of code. It is preferrable to use your editor to prepend each line in that code block with \inlinecode{//} as this cannot cause any problems. Most code editors have a feature that allows you to do this.

%TODO Possibly include note about Valadoc comments.

\section{Identifiers}
Identifiers are used for method and variable names. In the programmes we have written until now we have always written a \inlinecode{main()} method, in these methods the identifier is \inlinecode{main}. All the variable names in the second example were identifiers. Less obvious identifiers were \inlinecode{stdout} and \inlinecode{printf()}, these are two different identifiers separated by a separator - the period. Identifiers in Vala must can only contain the characters [A-Z], [a-z], [0-9] or an underscore, but the first character may not be a digit.

\section{Keywords}
%For now skip mention of using a reserved keyword by using @ until a proper explanation of the corner cases can be found.
Vala has a set of words that cannot be used as identifiers. These words are reserved for expressing information about the programme to the compiler.

%TODO Conver this to a Table
The list of reserved keywords is show in the following table:
\begin{center}
  \begin{tabular*}{0.9\textwidth}{@{\extracolsep{\fill}} l l l l l l l }
      if & else & switch & case  & default & do & while \\
      for & foreach & in & break & continue & return & try \\
      catch & finally & throw & lock & class & interface & struct \\
      enum & delegate & errordomain & const & weak & unowned & dynamic \\
      abstract & virtual & override & signal & extern & static & async \\
      inline & new & public & private & protected & internal & out \\
      ref & throws & requires & ensures & namespace & using & as \\
      is & in & new & delete & sizeof & typeof & this \\
      base & null & true & false & get & set & construct \\
      default & value & construct & void & var & yield & global \\
      owned & ~ & ~ & ~ & ~ & ~ & ~ \\
  \end{tabular*}
\end{center}

\section{Literals}
Literals are the constructs that represent values in the programme. An example of a constant here would be the value \inlinecode{true}, for a boolean, or \inlinecode{Varun} for a string. Here are some examples of literals:

\begin{itemize}
\item 1824
\item 1858.58
\item true
\item false
\item null
\item "Varun Madiath"
\end{itemize}

The first literal represents an integer, the next represents a floating-point value, the third and fourth are boolean constants, the fifth is the null constant, and the last represents a string. A literal can be substituted for a variable anywhere that a variable of its type is allowed. If you've actually read the entire list of keywords you would notice that the literals \inlinecode{true}, \inlinecode{false}, and \inlinecode{null} are also keywords.

\section{Separators}
In Vala, there are a few characters that are used as separators. The most commonly used separator in Vala is the semicolon. As you have seen, it is used to terminate
statements. The separators are shown in the following table:

%Try to fix this to automatically get wrapping at some point in time.
\begin{center}
  \begin{tabular*}{0.9\textwidth}{@{\extracolsep{\fill}} l l p{4in} }
    Symbol & Name & Purpose \\
    ( ) & Parentheses & They are used to contain the list of parameters in method definition and invocation. They are also used to define precedence  and contain expressions. Finally they are used to contain the data type when performing static type casting. \\
    % must use \relax to terminate expression. Else xelatex assumes the [] are defining an optional argument.
    \{ \} & Curly Braces & Used to contain the values of automatically initialised arrays. Also used to define a block of code, for classes, methods, and local scopes. \\ \relax
    [ ] & Brackets & Used to declare array types. Also used when dereferencing array values. \\
    ; & Semicolon & Terminates statements. \\
    , & Comma & Separates consecutive identifiers in a variable declaration. Also used to chain statements together inside a for statement. \\
    . & Period & Used to separate package names from subpackages and classes. Also used to separate a variable or method from a reference variable.
  \end{tabular*}
\end{center}
